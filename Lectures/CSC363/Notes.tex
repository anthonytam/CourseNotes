% Created 2018-03-19 Mon 11:45
% Intended LaTeX compiler: pdflatex
\documentclass[11pt]{article}
\usepackage[utf8]{inputenc}
\usepackage[T1]{fontenc}
\usepackage{graphicx}
\usepackage{grffile}
\usepackage{longtable}
\usepackage{wrapfig}
\usepackage{rotating}
\usepackage[normalem]{ulem}
\usepackage{amsmath}
\usepackage{textcomp}
\usepackage{amssymb}
\usepackage{capt-of}
\usepackage{hyperref}
\date{\today}
\title{}
\hypersetup{
 pdfauthor={},
 pdftitle={},
 pdfkeywords={},
 pdfsubject={},
 pdfcreator={Emacs 27.0.50 (Org mode 9.1.6)}, 
 pdflang={English}}
\begin{document}

\tableofcontents

\section{Week 2}
\label{sec:org9163c50}
\subsection{Pushdown Automata (PDA)}
\label{sec:org20c6ed0}
\begin{itemize}
\item L = \{ a\(^{\text{n}}\)b\(^{\text{n}}\) | n \(\ge\) 0 \}
\item 7 tuple, M = (Q, \(\sum\), \(\Gamma\), \(\delta\), s, F, \(\perp\))
\begin{itemize}
\item Q: Finite set of states
\item \(\sum\): Finite input alphabet
\item \(\Gamma\): finite stack alphabet
\item \(\delta\): transitions
\item \(\perp\) \(\in\) \(\Gamma\)
\item S \(\in\) Q
\item F \subseteq Q
\end{itemize}

\item \(\delta\) \subseteq (Q x (\(\sum\) U \{\(\epsilon\)\} x \(\Gamma\)) x (Q x \(\Gamma\)*)((p, a, A), (g, B\(_{\text{1}}\)B\(_{\text{2}}\)\ldots{}B\(_{\text{k}}\))
\begin{itemize}
\item When in state p, reading in "a", and A is on top of the stack, the machine \uline{may} move
to state g, pop A off the stack and push B\(_{\text{k}}\)B\(_{\text{k-1}}\)\ldots{}B\(_{\text{1}}\) (B\(_{\text{1}}\) is now on the stop of the stack.)
\end{itemize}
\end{itemize}
\subsubsection{Example}
\label{sec:org5b17cd4}
\begin{itemize}
\item Stage 1: Push A's
\item Stage 2: Pop A's
\item Stage 3: Are we accepting? (This mean B's and A's must be equal)
\end{itemize}
\subsubsection{More examples}
\label{sec:orga644066}
\begin{itemize}
\item a) L = \{ a\(^{\text{n}}\)b\(^{\text{n}}\)c\(^{\text{n}}\) | n \(\ge\) 0 \} (Impossible)
\item b) L = \{ ww\(^{\text{R}}\) | w \(\in\) \(\sum\)* \} (Possible)
\item c) L = \{ ww | w \(\in\) \(\sum\)* \} (Impossible)
\item d) L = \{ a\(^{\text{n}}\)b\(^{\text{m}}\)c\(^{\text{m}}\)d\(^{\text{n}}\) | n, m \(\ge\) 0 \} (Possible)
\end{itemize}
\subsection{Turing Machines (TM)}
\label{sec:orgbfdddc5}
\subsubsection{What can't PDAs do?}
\label{sec:orgfda6436}
\begin{itemize}
\item Infinite memory, but it's not "perfect"
\item What if we do have perfect/infinite memory
\item Does this allow for more accepted languages
\item How do we formally define it
\end{itemize}
\subsubsection{Definition}
\label{sec:orgca50c36}
\begin{itemize}
\item We'll say a TM is an 8-Tuple
\item M = \{ Q, \(\sum\), \(\Gamma\), \(\delta\), s, qaccept, qreject, \square \}
\begin{itemize}
\item qaccept \(\in\) Q
\item qreject \(\in\) Q
\item qaccept \(\ne\) qreject
\item \square \(\in\) \(\Gamma\)
\item \(\sum\) \subseteq \(\Gamma\) - \{ \square \}
\item \(\delta\): Qx\(\Gamma\) \(\rightarrow\) Q x \(\Gamma\) x \{L, R\}
\item \(\delta\)(g\(_{\text{1}}\), a) = (g\(_{\text{2}}\), b, R)
\begin{itemize}
\item If I am in g\(_{\text{1}}\) and it is an a, write a b and move right
\end{itemize}
\end{itemize}
\item Input is initally written on the tape
\item If at any point the TM enters state qaccept, it stops (halts) and accepts the input
\item If at any point the TM enters state qreject, it stops (halts) and rejects the input
\item Can be used as language accepters, or as output machines
\item Does all input get accepted or rejected?
\begin{itemize}
\item No, it can loop
\end{itemize}
\end{itemize}
\subsubsection{Example}
\label{sec:org635977f}
\begin{itemize}
\item Construct a TM M sich that L(M) = \{a\(^{\text{n}}\)b\(^{\text{n}}\)c\(^{\text{n}}\) | n \(\ge\) 0 \}
\item Start by scanning the input from L to R to ensure it's of the form a*b*c*
\item Scan from R to L, overwrite exactly one c, b, the a with a special character
\item Scan from L to R, overwrite exactly one a, b, then c repeat 2)
\item If at any time, you encouter a letter out of order, reject the input
\end{itemize}
\section{Week 3}
\label{sec:orgc2cc5e1}
\subsection{Delta function layout}
\label{sec:org76d6ef2}
(1, a, R) - Move to state 1, if its an A move right
\begin{center}
\begin{tabular}{rlllll}
 & a & b & c & - & \Box\\
\hline
S & (1, a, R) & Reject & Reject & / & Accept\\
1 & (1, a, R) & (2, b, R) & Reject & / & Reject\\
2 & Reject & (2, b, R) & (3, c, R) & / & Reject\\
3 & Reject & Reject & (3, c, R) & / & (4, \Box, L)\\
4 & Reject & Reject & (5, -, L) & (4, -, L) & Accept\\
5 &  &  &  &  & \\
6 &  &  &  &  & \\
7 &  &  &  &  & \\
8 &  &  &  &  & \\
9 &  &  &  &  & \\
10 &  &  &  &  & \\
11 &  &  &  &  & \\
\end{tabular}
\end{center}
\subsection{Output Turing Machine}
\label{sec:org5378a82}
\subsubsection{Example}
\label{sec:orgb019ba1}
Let \(\sum\) = \{a\}. write a TM such that when M halts, the tape contains ww where w is the input. (Double the a's)
\begin{itemize}
\item a = aa
\item aaa = aaaaaa
\end{itemize}
\subsection{Language of a Turing Machine}
\label{sec:orgef5dfdf}
\begin{itemize}
\item The collection or set of strings which M accepts L(M)
\item A language L' is Turing-recognizable if \(\exists\) a Turing machine M, L'=L(M)
\begin{itemize}
\item w \(\in\) L' \(\Rightarrow\) if we run M on w, it accepts w
\item w \(\notin\) L' \(\Rightarrow\) if we run M on w, it rejects or infinite loops
\end{itemize}
\item A language L\(^{\text{-}}\) is Turing Decidable if \(\exists\) a TM, M sich L\(^{\text{-}}\) = L(M) and M halts on all input
\begin{itemize}
\item w \(\in\) L\(^{\text{-}}\) \(\Rightarrow\) running M on w accepts
\item w \(\notin\) L\(^{\text{-}}\) \(\Rightarrow\) running M on w rejects
\end{itemize}
\end{itemize}
\subsubsection{Question}
\label{sec:org8a66615}
\begin{itemize}
\item L is recognizable \(\Leftarrow\) L is decidable
\item L is decidable \(\Rightarrow\) \(\forall\) turing machines M, such that L' = L(M) \(\Rightarrow\) if M is run on w \(\in\) L, M will halt? \textbf{NO}
\begin{enumerate}
\item Let M be a TM which decided L'
\item Make a new TM, M', which mimics M except when M rejects, M' will loop
\end{enumerate}
\end{itemize}
\subsection{Multi-tape Turing Machines}
\label{sec:orga99c884}
\begin{itemize}
\item Just like a classic TM, but it has k tapes
\item \(\delta\): Q x \(\Gamma\) \(\rightarrow\) Q X \(\Gamma\) x \{L, R\} is replaced with \(\delta^{\text{k}}\): Q X \(\Gamma^{\text{k}}\) \(\rightarrow\) Q X \(\Gamma^{\text{k}}\) X \{L, R\}\(^{\text{k}}\)
\end{itemize}
\subsubsection{Show a single tape TM can simulate a multi-tape TM}
\label{sec:org77ee174}
\begin{itemize}
\item Encode multiple tapes on a single tape
\end{itemize}
\(\Gamma\) = \{a, b, \Box\} \(\rightarrow\) \(\Gamma\)' = \{a, b, a', b', \Box\}
\subsection{Single infinite tape TM}
\label{sec:orga71018a}
\begin{itemize}
\item Same as a classic TM but infinite only in one direction
\item Just add some left padding
\item This is like a 2 tape TM, the 2nd tape is the 1st one flipped
\end{itemize}
\subsection{Enumerators}
\label{sec:org6dfe615}
\begin{itemize}
\item 2 tapes and a control
\begin{itemize}
\item Control has the output tape with the following conditions
\begin{itemize}
\item Write Only
\item Tape alphabet = \(\sum\)
\item Only works to the right
\end{itemize}
\item Control has the work tape with the following confitions
\begin{itemize}
\item read/write
\item Basically a TM
\item No accept or reject states
\item Enumerate state
\end{itemize}
\end{itemize}
\begin{itemize}
\item Whenever the control reaches the enumerate state, it prints out whatever is on the output tape and clears it
\item L(E\_ is the set of strings which W will eventually print out
\end{itemize}
\end{itemize}
\section{Week 4}
\label{sec:orga312c06}
\begin{itemize}
\item I slept\ldots{}
\end{itemize}
\section{Week 4 Tutorial}
\label{sec:org1bd9600}
\subsection{Diagnolization to prove R does not map to N (Not countable)}
\label{sec:org705afeb}
\begin{itemize}
\item d\(_{\text{11}}\)d\(_{\text{12}}\)d\(_{\text{13}}\)\ldots{}d\(_{\text{1n}}\)
d\(_{\text{21}}\)\ldots{}\ldots{}..d\(_{\text{2n}}\)
\ldots{}\ldots{}\ldots{}..
d\(_{\text{n1}}\)\ldots{}\ldots{}..d\(_{\text{nn}}\)
\item Construct e such that e\(_{\text{ii}}\) = (d\(_{\text{ii}}\) + 2) \% 10
\end{itemize}
\subsection{Proof that Eq\(_{\text{BFA}}\) = <(DFA1, DFA2); <(DFA1)=<(DFA2)>}
\label{sec:org09f4b57}
\begin{itemize}
\item 2\(\cdot\)L(DFA1) = L(DFA2) \(\Rightarrow\) (L(DF1) \intersect L(DFA2)) = \(\epsilon\)
AND (L(DFA1) \intersect L(DFA2)) = \(\epsilon\)
\item Checking if L(DFA) = \(\epsilon\) is Decidable
\begin{itemize}
\item Go through every state in the DFA
\item If you go to any state that is acccepting with a string that is not empty, reject
\begin{itemize}
\item Otherwise accept.
\end{itemize}
\end{itemize}
\item Full proof on pg 169 of tb
\begin{itemize}
\item Introduction to theroty of computation 2\(^{\text{nd}}\) edition By M. Sipser
\end{itemize}
\end{itemize}
\section{Week 5}
\label{sec:org07c6160}
\subsection{Continuing..}
\label{sec:org5982687}
\begin{itemize}
\item A\(_{\text{TM}}\) = \{M\#\(_{\text{W}}\) | M accepts w \}
\begin{itemize}
\item The membership/acceptance problem
\item Thm: A\(_{\text{TM}}\) is a recognizable language
\item Proof: Construct a new TM U
\begin{itemize}
\item U = "on input M\#\(_{\text{w}}\) where M is a TM and w is a string"
\begin{enumerate}
\item Simulate M running on w
\item If M accepts, U accepts
\item If M rejects, U rejects
\end{enumerate}
\end{itemize}
\item Universal Turing Machine (UTM)
\item Thm: M is undecideable
\item Proof: Assume that A\(_{\text{TM}}\) is decideable
\(\Rightarrow\) there exost a TM D
D(M\#\(_{\text{w}}\)) = \{ accept if M accepts W, reject is M does not accept W \} 
P = "on input M, where M is a TM:"
\begin{enumerate}
\item Run D on input M\#\(_{\text{M}}\)
\item Output the opposite of D
\begin{itemize}
\item if D accepts, reject
\item if D rejects, accept
\end{itemize}
\item P(M) = \{ accept if M does not accept M, reject if M does accept M \}
\end{enumerate}
\begin{itemize}
\item What if we run P on itself?
P(P) = \{ accept is P rejects, reject if P accepts \}
  This is a paradox \(\Rightarrow\) Contradiction
  \(\therefore\) A\(_{\text{TM}}\) is undecideable
\end{itemize}
\end{itemize}
\item Def Lbar = \(\sum\)* - L
\begin{itemize}
\item Thm: A\(_{\text{TM}}\)Bar is unrecognizeable
\item Proof: Will follow from 
\begin{itemize}
\item Thm: A is decidable iff A and ABar are recognizable
\item Proof: 
\begin{enumerate}
\item Show A is decideable \(\Rightarrow\) A and ABar are recognizable
\item Show A is decideable \(\Rightarrow\) A is recognizable
\item \(\Leftarrow\) A is recognizable \(\Rightarrow\) \(\exists\) TM m such that \(\forall\)w, M accept (and halt) w iff w \(\in\) A
\end{enumerate}
\end{itemize}
\begin{itemize}
\item Let M\(_{\text{1}}\) and M\(_{\text{2}}\) recognize A and ABar respectively \(\Rightarrow\) A\(_{\text{W}}\) M\(_{\text{1}}\) halts if w\(\in\)A, M\(_{\text{2}}\) halt is w\(\notin\)A
\item M = "Run M\(_{\text{1}}\) and M\(_{\text{2}}\) in parallel"
\begin{enumerate}
\item If M\(_{\text{1}}\) accepts, accept
\item If M\(_{\text{2}}\) accepts, reject
\end{enumerate}
\item M is a decider for A \(\Rightarrow\) A is decideable
\end{itemize}
\item Thm (again): A\(_{\text{TM}}\)Bar is unrecognizeable
\begin{itemize}
\item if A\(_{\text{TM}}\)Bar was recognizable \(\Rightarrow\) A\(_{\text{TM}}\) is decidable, which it isn't
\end{itemize}
\end{itemize}
\end{itemize}
\subsection{Halting Problem}
\label{sec:org2b04cb3}
\subsubsection{Turings Method}
\label{sec:orge0fbaca}
\begin{itemize}
\item HP = \{ M\#\(_{\text{W}}\) | M halts, on W \}
\item Thm: The halting problem is undecideable
\item Proof: 
\begin{enumerate}
\item \(\sum\)* is countable
\item The set of all TMs is countable
\item Construct a TM P:
P(w) = "on input w, construct M\(_{\text{w}}\) and we run D on M\(_{\text{w\#w}}\)
\begin{enumerate}
\item If D rejects, accept
\item if D accepts, loop
\end{enumerate}
\item Assume HP is decideable
\begin{itemize}
\item D(m\#\(_{\text{W}}\)) = \{ accept if M halts on w, reject if M loops on w \}
\item P is not in my table, \(\therefore\) contraduction.
\end{itemize}
\end{enumerate}
\end{itemize}
\subsubsection{Method 2}
\label{sec:orgefb8c32}
\begin{itemize}
\item Assume HP is decideable
\begin{itemize}
\item Then \(\exists\) a decider for HP
\end{itemize}
D\(_{\text{1}}\)(M\#\(_{\text{w}}\)) = \{ "accept if M halts on w, reject if M loops on w" \}
D\(_{\text{2}}\) = "on input M\#\(_{\text{w}}\), M is a TM, w is a string"
\begin{enumerate}
\item Run D\(_{\text{1}}\) on input M\#\(_{\text{w}}\)
\item If D\(_{\text{1}}\) rejects, reject
\item If D\(_{\text{1}}\) accepts, simulate M on w
3.1) If M accepts, accept
3.2) If M rejects, reject
\end{enumerate}
D\(_{\text{2}}\)(M\#\(_{\text{w}}\)) = \{ "if M loops on w reject, if M rejects w reject, if M accepts w accept" \}
D\(_{\text{2}}\) decides A\(_{\text{TM}}\)
A\(_{\text{TM}}\) = \{ M\#\(_{\text{w}}\) | M accepts w \}
Contradiction, \(\therefore\) HP is undecideable
\end{itemize}
\section{Week 6}
\label{sec:org5aedea6}
\begin{itemize}
\item Sick
\end{itemize}
\section{Week 7}
\label{sec:org55a1c79}
\begin{itemize}
\item Last Time
\begin{itemize}
\item A \(\le_{\text{m}}\) B
\item ES and REG are both undecideable
\end{itemize}
\item Given a TM M
\begin{enumerate}
\item Accepts any string
\item Accepts the string OOOO
\item Accepts every string
\item Accepts a finite set
\item Accepts a CFL
\item Has 100 or more states
\item Has more then 100 steps on input w
\item 1 through 5 are undecideable
\begin{itemize}
\item About the language of the TM (L(M))
\end{itemize}
\item 6 through 7 are decideable
\begin{itemize}
\item About the TM (M)
\end{itemize}
\end{enumerate}
\end{itemize}
\subsection{Anything about the language is undecideable}
\label{sec:orge5f70ee}
\begin{itemize}
\item Let S denote the set of all languages
\item Let a property P be a subset of S
\item A TM M has the property P if L(M)\(\in\)P
\item Property
\begin{itemize}
\item A TM accepts no strings \(\Rightarrow\) P\(_{\text{empty}}\) = \{ \(\empty\) \}
\item A property is non trivial if \(\exists\) M\(_{\text{1}}\) and M\(_{\text{2}}\) such that L(M\(_{\text{1}}\)) \(\in\) P and L(M\(_{\text{2}}\)) \(\notin\) P
\item P\(_{\text{trivial}}\) = \{ L | L is recognizable \}
\end{itemize}
\item Rice's Thm:
\begin{itemize}
\item Let P be a non-trivial property of languages of Turing Machines
\begin{itemize}
\item L\(_{\text{p}}\) = \{ M | L(M) \(\in\) P \} is undecideable
\end{itemize}
\end{itemize}
\end{itemize}
\subsection{Midterm}
\label{sec:org62cf339}
\begin{itemize}
\item Thursday March 1\(^{\text{st}}\), 6:00-8:00pm
\item IB120
\item 5 Questions
\item Topics
\begin{itemize}
\item PDAs
\item Low Level TMs
\begin{itemize}
\item Define a TM, states, transitions, alphabet, etc
\end{itemize}
\item Models of Equivalence
\begin{itemize}
\item Ignore enumerators
\end{itemize}
\item Decideability and Recognizeability
\item A\(_{\text{TM}}\)/Halting Problems proofs
\begin{itemize}
\item Don't memorize the proof in full
\item Know the idea, fill in the blanks/find the error
\end{itemize}
\item Reductions
\begin{itemize}
\item Definitions
\item Proofs 
\begin{itemize}
\item Show that language L is decideable/recognizeable using a reduction
\end{itemize}
\end{itemize}
\end{itemize}
\item Define L is recognizable
\(\exists\) a TM M such that \(\forall\)w if w\(\in\)L then M accepts w and if w\(\notin\)L M loops or rejects w
\item HP = \{ M\(_{\text{\#w}}\) | M halts on w \}
\begin{itemize}
\item R(M\(_{\text{\#w}}\)) = run M on w, if M accepts \(\rightarrow\) accept. If M rejects \(\rightarrow\) accept. If M loops \(\rightarrow\) loop
\item M\(_{\text{\#w}}\) \(\in\) HP iff M doesn't loop on w
\end{itemize}
\item A is recognizable \textlnot{}\(\rightarrow\) A' is unrecognizeable
\item A is undecideable and A is recognizable \(\rightarrow\) A' is unrecognizeable
\end{itemize}
\subsubsection{Past test, quesiton 7}
\label{sec:org3990c38}
\begin{itemize}
\item L'' = \{ M : L(M) = L\(_{\text{L}}\) \}
\item L\(_{\text{L}}\) = \{ O\(^{\text{n}}\)l\(^{\text{m}}\) | n < m \}
\begin{itemize}
\item Oll \(\in\) L\(_{\text{L}}\)
\item O \(\notin\) L\(_{\text{L}}\)
\item \(\epsilon\) \(\notin\) L\(_{\text{L}}\)
\item lO \(\notin\) L\(_{\text{L}}\)
\end{itemize}
\item is L'' decideable, recognizable?
\item is L''' decidable, recognizable?
\item HP, A\(_{\text{TM}}\), HP', A\(_{\text{TM}}\)'
\item HP' \(\le\) mL
\item Show L'' is undecideable
\begin{itemize}
\item HP \(\le_{\text{m}}\) L''
\item f(M\(_{\text{\#w}}\))\(\in\)L'' \(\Leftarrow \Rightarrow\) M\(_{\text{\#W}}\) \(\in\) HP
\item f(M\(_{\text{\#W}}\)) = on input x, set x aside
\begin{itemize}
\item Run M on W
\item run a L\(_{\text{L}}\)\(_\)\(_\) recognizer on x
\begin{itemize}
\item if it accepts \(\rightarrow\) accept
\item if it rejcts \(\rightarrow\) reject
\end{itemize}
\end{itemize}
\item L(f(M\(_{\text{\#W}}\))) = \{ L\(_{\text{L}}\) if M halts on w, \(\empty\) if M loops on W \}
\item f(M\(_{\text{\#W}}\)) \(\in\) L'' \(\Leftarrow \Rightarrow\) L(f(M\(_{\text{\#W}}\))) = L\(_{\text{L}}\) \(\Leftarrow \Rightarrow\) M halts on W \(\Leftarrow \Rightarrow\) M\(_{\text{\#W}}\) \(\in\) HP
\item \(\therefore\) L''in undecideable
\end{itemize}
\item if A \(\le_{\text{m}}\) B then A' \(\le_{\text{m}}\) B'
\begin{itemize}
\item HP' \(\le_{\text{m}}\) L'''
\item \(\therefore\) L''' is unrecognizeable
\end{itemize}
\end{itemize}
\section{Week 8}
\label{sec:org7dac399}
\begin{itemize}
\item What can (or can't) camputers do (effectivness)
\item So something using (relatively) little resources
\begin{itemize}
\item Time and space
\end{itemize}
\end{itemize}
\subsection{Moving Forward}
\label{sec:orgb49512e}
\begin{itemize}
\item Look at low level TM
\item Compare Models
\item Classify problems / Heirachy - Reductions
\end{itemize}
\subsection{The Lecture}
\label{sec:org49f8ac3}
\begin{itemize}
\item Def
Let M be a total deterministic TM, the time complexity of M is a function f:N \(\rightarrow\) N, f(n) is the max number of steps M
uses on any input of length n
\begin{itemize}
\item M is a f(n) TM
\item M runs in f(n) time
\item Def
f, g N\(\rightarrow\)R\(^{\text{T}}\), f(n) = O(g(n)) if \(\exists\)n\(_{\text{0}}\), c \(\forall\) n>n\(_{\text{0}}\): f(n) < cg(n)
\end{itemize}
\item Def
f, g: N\(\rightarrow\)R* f(n) = o(g(n)) if $$lim \frac{f(n)}{g(n)} = 0$$\\
g grows \uline{faster} then f eventually
\item Facts
\begin{itemize}
\item f(n) = O(f(n))
\item f(n) \(\ne\) o(f(n))
\item f(n) = O(g(n)) \textlnot{}\(\rightarrow\) f(n) = o(g(n))
\item f(n) = o(g(n)) \(\rightarrow\) f(n) = O(g(n))
\end{itemize}
\item Consider the language L = \{0\(^{\text{n}}\)\(^\)\(^\)1\(^{\text{n}}\) | n\(\ge\)0 \}
M\(_{\text{1}}\) = on input w
\begin{enumerate}
\item Scan from L to R
\item Cross off first O encountered
2,1) Then cross off the first 1 encountered
\item When \box is reached, go back to the begining
\item Repeat 2
4,1) If anything is found out of place, reject
4,2) If no 0 or 1 before \box, accept
\end{enumerate}
\item M is a f(n) = o(nlog(n)), then L(M), is regular 
\begin{itemize}
\item Def: t: N\(\rightarrow\)R\(^{\text{t}}\), the time class TIME(t(n)) is the set of all languages that are decidable by a O(t(n)) TM.
\begin{itemize}
\item is L \(\in\) TIME(N\(^{\text{3}}\))?
\item is L \(\in\) TIME(N)?
\item is L \(\in\) TIME(nlog(n))
\end{itemize}
\item Thm
Let t(n) be a function where t(n)\(\ge\)n. Then every t(n) time multitape TM has an equivalent O(t\(^{\text{2}}\)(n)) single tape TM
\begin{itemize}
\item How long can the single tape be at any moment in time? Assume t(n)\(\ge\)n
\begin{itemize}
\item Each multitape can be at most t(n) long
\item The single table can be at \(\le\) k*t(n) + k - 1 (Account for the boundry characters) = O(t(n))
\item In the worst case for a single "step" take
\[ O(t(n)) + k shifts \]
\[ O(t(n)) + k*O(t(n)) \]
\[ O(t(n)) \]
\item Number of steps to simulate = t(n)
\item \(\therefore\) Total simulation time t(n)*O(t(n)) = O(t\(^{\text{2}}\)(n))
\item Let N be a total NTM, its running time f(n) is the running time of itslongest branch
\item Thm
Let t(n)\(\ge\)n, then every t(n) NTM has an equivalent 2\(^{\text{O(t(n))}}\) time deterministic TM
\item Def
P is the class of languages that are decideable in polynomial time on a deterministic single-tape TM. 
\[ P = U^{inf}_{k = 0} TIME(n^k) \]
\end{itemize}
\end{itemize}
\end{itemize}
\end{itemize}
\section{Week 9}
\label{sec:orgf59edb8}
\begin{itemize}
\item All deterministic computational models are polynomial equivalent
\begin{itemize}
\item If you can code a program which runs in polynomial time, a single table TM exists, which does the same thing
\end{itemize}
\end{itemize}
\subsection{Example}
\label{sec:orgf5a6317}
PATH = \{ <G, s, t> | There's a path from s to t in G \}
\begin{itemize}
\item Theorem
PATH \(\in\) P 
\begin{itemize}
\item Proof
A poly time alporithm M operates as follows
M(G, s, t) = 
\begin{itemize}
\item "Mark" s
\item While at least one more node has been marked since last iteration
\begin{itemize}
\item For each edge (a, b) if a is marked and b is not, mark b
\end{itemize}
\begin{itemize}
\item If t is marked \(\rightarrow\) accept, else reject
\end{itemize}
\end{itemize}
\end{itemize}
\end{itemize}
\subsection{Example}
\label{sec:orgdc6ccdb}
HPATH = \{ <G, s, t> | there's a Hamiltonian path from s to t \}
\begin{itemize}
\item Def
A Hamiltonian path goes through every other node \uline{exactly} once
\item Def
A verifier for a language L is an algorithm V where:
\begin{itemize}
\item L = \{ w | V accepts <w, c> for some c \}
\item c is called the certificate
\item A poly time verified runs in polynomial time in the length of w
\item A language L is polynomial time verifiable if it has a polynomial time verifier (This TM always halts)
\end{itemize}
\item Def One
NP is the class of languages which have polynomial time verifiers
\item Def Two
A language is in NP iff it is decided by some polynomial time NTM
\end{itemize}
\subsection{Question}
\label{sec:org853a3dc}
If a language is in P, it is also in NP?
\begin{itemize}
\item Def One
\begin{itemize}
\item V(<w, c>) = run D\(_{\text{poly}}\) on w
\item D\(_{\text{Poly}}\) = decider for L
\end{itemize}
\item Def Two
\begin{itemize}
\item \{ TM \} \subseteq \{NTM\}
\end{itemize}
\end{itemize}
\subsection{Question}
\label{sec:orgd2687e3}
If a language is in NP, it is also in P?
\begin{itemize}
\item Well this is the P = NP problem. We know sometimes, not if it is always.
\end{itemize}
\subsection{Complexity Reductions}
\label{sec:org1aad090}
\begin{itemize}
\item Def
Language A is a polynomial time reduceable to language B, A \(\le_{\text{p}}\) B, if \(\exists\) a polynomial time computable function f, 
where \(\forall\)w, w \(\in\) A \(\Leftarrow \Rightarrow\) f(w) \(\in\) B
\begin{itemize}
\item Typical use
Given problem A, assume we have a black box which solces problem B. In polynomial time, make your instance of A
look like an instrance of problem B. Run your B black box on that instance, return an answer appropriatly
\end{itemize}
\item Example
LONG = \{ <G, s, t, d> | \(\exists\) a path from s to t in G, with at least distance of d \} 
\begin{itemize}
\item No cyles allowed, can't visit a node more then once
\item Prove that HPATH \(\le_{\text{p}}\) LONG
\begin{enumerate}
\item Assume I have a black box which solved LONG
\item To solve HPATH(G, s, t)
\begin{enumerate}
\item Set all edge weights to 1
\item Call LongSol(G, s, t, |v| - 1)
\begin{itemize}
\item If yes \(\rightarrow\) yes, if no \(\rightarrow\) no
\end{itemize}
\end{enumerate}
\end{enumerate}
\end{itemize}
\begin{itemize}
\item Thm
If A \(\le_{\text{P}}\) B, and A \(\in\) P, then A \(\in\) P
\item Caollary
If A \(\le_{\text{p}}\) B and A \(\notin\) P, then B \(\notin\) P
\end{itemize}
\end{itemize}
\section{Week 10}
\label{sec:orge06c4de}
\subsection{From last time}
\label{sec:org5e61a02}
\begin{itemize}
\item A \(\le_{\text{p}}\) B
\begin{itemize}
\item If we can solve B, then we can solve A
\item Then B is at least as hard to solve as A is
\end{itemize}
\end{itemize}
\subsection{Toolbox of problems}
\label{sec:orga9c8542}
\begin{itemize}
\item Vertex Cover
Given a graph G=(V, E), a Vertex Cover, C, is a set of verticies such that \(\forall\)(v\(_{\text{1}}\), v\(_{\text{2}}\))\(\in\) E, v\(_{\text{1}}\) \(\in\) C, v\(_{\text{2}}\) \(\in\) C\\
VertexCover = \{(G, K) | G has a vertex cover of size k or less \}
\item Independent-Set
Given a graph G=(V, E), an independent set is a set of verticies I, such that \(\forall\)(v\(_{\text{1}}\), v\(_{\text{2}}\)) \(\in\) E, \textlnot{}(v\(_{\text{1}}\) \(\in\) I and v\(_{\text{2}}\) \(\in\) I)\\
Independent-Set = \{(G, K) | G has an independent set of size K \}
\begin{itemize}
\item Prove: Independent Set \(\le_{\text{P}}\) Vertex Cover
Let C be a vector cover of G=(V, E). Consider any two nodes v\(_{\text{1}}\) and v\(_{\text{2}}\) in V-C.\\
What if v\(_{\text{1}}\), v\(_{\text{2}}\) \(\in\) V-C and (v\(_{\text{1}}\), v\(_{\text{2}}\))\(\in\) E \(\Rightarrow\) C is not a VC
\begin{itemize}
\item \(\therefore\) v\(_{\text{1}}\), v\(_{\text{2}}\) \(\in\) V-C we know (v\(_{\text{1}}\), v\(_{\text{2}}\))\(\notin\) E
\item \(\therefore\) V-C is an independent set in G
\item \(\therefore\) If there is a vertex cover of size |V| - k in G, then there's an independent set of size k in G
\end{itemize}
IS(G, K) = call VertexCoverSol(G, |V|-k)
\begin{itemize}
\item If accept \(\rightarrow\) accept (yes)
\item If reject \(\rightarrow\) reject (no)
\end{itemize}
\end{itemize}
\item Clique
Given a graph G=(V, E), a cliquw of the graph is a set S such that S\subseteq{}V and \(\forall\) v\(_{\text{1}}\), v\(_{\text{2}}\) \(\in\) S, 
(v\(_{\text{1}}\), v\(_{\text{2}}\)) \(\in\) E or v\(_{\text{1}}\) = v\(_{\text{2}}\).
\end{itemize}
\end{document}
